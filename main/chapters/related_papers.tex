\section{Related papers}
The problem this work focuses on solving is the deployment of generated algorithms in the real world.
The current technical issues faced when trying to implement such a solution are related to the choice of the parameters for a given computer program, and scoring the quality of the produced computer programs within the simulation due to model mismatch.
Our specific contributions, to that end, are: multi-shot local policy search, training and test environments, and a method that utilises a UAV platform to test generated control algorithms.


In the literature, the most related paper to our work appears to be `LLM-Assisted Local Search for Policy Synthesis' \cite{sadmine2024language}.
With Sadamine, Baier, and Lelis's work, they demonstrate how LLMs can be used to create programmatic policies for various tasks in a single shot.
Notably, our work differs from theirs in two separate instances: our policies are tested in the real world, and we focus on multiple queries to the LLM for program synthesis.
Based on our previous work, it appears that a wide approach to program synthesis has provided better results than a small initial population for performing the local search.

In terms of evolutionary papers related to ours, the basis for this paper is FunSearch \cite{romera2024mathematical}.
Within FunSearch, the authors demonstrated how evolutionary queries can be utilised to surpass human performance on combinatorial problems.
Compared to our work, which focuses on both evolutionary and local parameterisation, the FunSearch method is costly and yields significantly worse results.
The primary reasons behind the increased cost of FunSearch are related to local parameterisation.
It is unlikely that the FunSearch technique, on its own, can be used to create a high-quality control system within a reasonable time or without incurring significant excess computational cost. The inefficiencies of FunSearch are due to the choice of coefficient values within program synthesis. Opposed to simply altering a variable value, the FunSearch technique typically opts for a more complex solution to a problem.

Li et al.\cite{li2024guiding} focused on an enumerative search technique for program synthesis.
Li et al.'s work focuses primarily on the synthesis of programs for solving a complex dataset (SyGuS), with a continuous bi-directional communication feedback loop for failed programs.
The primary contribution of their research is the creation of a `probabilistic Context-Free Grammar (pCFG)', a list of failed programs, to help the LLM achieve a correct solution.
A similar technique is used in our methodology.
Opposed to giving a specific name to this technique, we define how and why we implemented it.
For our work, we instead focus on the implicit information provided to the LLM through evolutionary queries and epigenetic program selection.
Our evolution queries include previous examples of programs, along with a meta-heuristic, the `score' of a program.
Theoretically, providing the programs in the query as our approach, as opposed to theirs, provides increased context for the model to make intelligent decisions.
Our work, rather than focusing on achieving a broad application of the technique, aims to apply the optimisation to a single use case.
This allows us to incorporate information easily and include meta-heuristic information into our queries—a difficult task to achieve in the broad sense.

A related work that might help support the theoretical results of this approach, which has not been implemented, is the usage of multimodality within query generation before program synthesis.
For example, Huang et al. \cite{huang2024multimodal} performed work utilising GPT-4's multimodal features, demonstrating how multimodality can enhance optimisation performance.
In the context of our work, implementing this makes sense simply due to the ease and clarity with which information about a trajectory can be displayed in a visual format.
However, a decision was made not to incorporate multimodal queries within our results, primarily due to our incentive to keep the monetary cost of generation at zero.
We believe that ensuring the optimisation process is free increases much-appreciated scrutiny and facilitates widespread adoption.

For a recent review and survey of papers covering the relevant topic area, the authors recommend either Wu et al. \cite{wu2024evolutionary} or Huang et al. \cite{huang2024large}. Both papers provide a comprehensive overview of the current progress in Large Language Models and Optimisation.

The reasoning behind this decision is that PSO has proven effective on a range of control problems. For example, researchers have applied PSO to automatically tune PID gains for UAV attitude and position control, achieving improved trajectory tracking performance without manual tuning \cite{rinaldi2025pso}.


To our knowledge, no one has yet deployed a generated algorithm in the real world.
